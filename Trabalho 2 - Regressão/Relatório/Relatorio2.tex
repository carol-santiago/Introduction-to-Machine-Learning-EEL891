\documentclass[11pt]{article}

    \usepackage[breakable]{tcolorbox}
    \usepackage{parskip} % Stop auto-indenting (to mimic markdown behaviour)
    \usepackage[T1]{fontenc}
    \usepackage[utf8]{inputenc}

    % Define a font que você deseja usar
     \usepackage{palatino}

    % Basic figure setup, for now with no caption control since it's done
    % automatically by Pandoc (which extracts ![](path) syntax from Markdown).
    \usepackage{graphicx}
    % Maintain compatibility with old templates. Remove in nbconvert 6.0
    \let\Oldincludegraphics\includegraphics
    % Ensure that by default, figures have no caption (until we provide a
    % proper Figure object with a Caption API and a way to capture that
    % in the conversion process - todo).
    \usepackage{caption}
    \DeclareCaptionFormat{nocaption}{}
    \captionsetup{format=nocaption,aboveskip=0pt,belowskip=0pt}

    \usepackage{float}
    \floatplacement{figure}{H} % forces figures to be placed at the correct location
    \usepackage{xcolor} % Allow colors to be defined
    \usepackage{enumerate} % Needed for markdown enumerations to work
    \usepackage{geometry} % Used to adjust the document margins
    \usepackage{amsmath} % Equations
    \usepackage{amssymb} % Equations
    \usepackage{textcomp} % defines textquotesingle
    % Hack from http://tex.stackexchange.com/a/47451/13684:
    \AtBeginDocument{%
        \def\PYZsq{\textquotesingle}% Upright quotes in Pygmentized code
    }
    \usepackage{upquote} % Upright quotes for verbatim code
    \usepackage{eurosym} % defines \euro

    \usepackage{iftex}
    \ifPDFTeX
        \usepackage[T1]{fontenc}
        \IfFileExists{alphabeta.sty}{
              \usepackage{alphabeta}
          }{
              \usepackage[mathletters]{ucs}
              \usepackage[utf8x]{inputenc}
          }
    \else
        \usepackage{fontspec}
        \usepackage{unicode-math}
    \fi

    \usepackage{fancyvrb} % verbatim replacement that allows latex
    \usepackage{grffile} % extends the file name processing of package graphics
                         % to support a larger range
    \makeatletter % fix for old versions of grffile with XeLaTeX
    \@ifpackagelater{grffile}{2019/11/01}
    {
      % Do nothing on new versions
    }
    {
      \def\Gread@@xetex#1{%
        \IfFileExists{"\Gin@base".bb}%
        {\Gread@eps{\Gin@base.bb}}%
        {\Gread@@xetex@aux#1}%
      }
    }
    \makeatother
    \usepackage[Export]{adjustbox} % Used to constrain images to a maximum size
    \adjustboxset{max size={0.9\linewidth}{0.9\paperheight}}

    % The hyperref package gives us a pdf with properly built
    % internal navigation ('pdf bookmarks' for the table of contents,
    % internal cross-reference links, web links for URLs, etc.)
    \usepackage{hyperref}
    % The default LaTeX title has an obnoxious amount of whitespace. By default,
    % titling removes some of it. It also provides customization options.
    \usepackage{titling}
    \usepackage{longtable} % longtable support required by pandoc >1.10
    \usepackage{booktabs}  % table support for pandoc > 1.12.2
    \usepackage{array}     % table support for pandoc >= 2.11.3
    \usepackage{calc}      % table minipage width calculation for pandoc >= 2.11.1
    \usepackage[inline]{enumitem} % IRkernel/repr support (it uses the enumerate* environment)
    \usepackage[normalem]{ulem} % ulem is needed to support strikethroughs (\sout)
                                % normalem makes italics be italics, not underlines
    \usepackage{mathrsfs}
    

    
    % Colors for the hyperref package
    \definecolor{urlcolor}{rgb}{0,.145,.698}
    \definecolor{linkcolor}{rgb}{.71,0.21,0.01}
    \definecolor{citecolor}{rgb}{.12,.54,.11}

    % ANSI colors
    \definecolor{ansi-black}{HTML}{3E424D}
    \definecolor{ansi-black-intense}{HTML}{282C36}
    \definecolor{ansi-red}{HTML}{E75C58}
    \definecolor{ansi-red-intense}{HTML}{B22B31}
    \definecolor{ansi-green}{HTML}{00A250}
    \definecolor{ansi-green-intense}{HTML}{007427}
    \definecolor{ansi-yellow}{HTML}{DDB62B}
    \definecolor{ansi-yellow-intense}{HTML}{B27D12}
    \definecolor{ansi-blue}{HTML}{208FFB}
    \definecolor{ansi-blue-intense}{HTML}{0065CA}
    \definecolor{ansi-magenta}{HTML}{D160C4}
    \definecolor{ansi-magenta-intense}{HTML}{A03196}
    \definecolor{ansi-cyan}{HTML}{60C6C8}
    \definecolor{ansi-cyan-intense}{HTML}{258F8F}
    \definecolor{ansi-white}{HTML}{C5C1B4}
    \definecolor{ansi-white-intense}{HTML}{A1A6B2}
    \definecolor{ansi-default-inverse-fg}{HTML}{FFFFFF}
    \definecolor{ansi-default-inverse-bg}{HTML}{000000}

    % common color for the border for error outputs.
    \definecolor{outerrorbackground}{HTML}{FFDFDF}

    % commands and environments needed by pandoc snippets
    % extracted from the output of `pandoc -s`
    \providecommand{\tightlist}{%
      \setlength{\itemsep}{0pt}\setlength{\parskip}{0pt}}
    \DefineVerbatimEnvironment{Highlighting}{Verbatim}{commandchars=\\\{\}}
    % Add ',fontsize=\small' for more characters per line
    \newenvironment{Shaded}{}{}
    \newcommand{\KeywordTok}[1]{\textcolor[rgb]{0.00,0.44,0.13}{\textbf{{#1}}}}
    \newcommand{\DataTypeTok}[1]{\textcolor[rgb]{0.56,0.13,0.00}{{#1}}}
    \newcommand{\DecValTok}[1]{\textcolor[rgb]{0.25,0.63,0.44}{{#1}}}
    \newcommand{\BaseNTok}[1]{\textcolor[rgb]{0.25,0.63,0.44}{{#1}}}
    \newcommand{\FloatTok}[1]{\textcolor[rgb]{0.25,0.63,0.44}{{#1}}}
    \newcommand{\CharTok}[1]{\textcolor[rgb]{0.25,0.44,0.63}{{#1}}}
    \newcommand{\StringTok}[1]{\textcolor[rgb]{0.25,0.44,0.63}{{#1}}}
    \newcommand{\CommentTok}[1]{\textcolor[rgb]{0.38,0.63,0.69}{\textit{{#1}}}}
    \newcommand{\OtherTok}[1]{\textcolor[rgb]{0.00,0.44,0.13}{{#1}}}
    \newcommand{\AlertTok}[1]{\textcolor[rgb]{1.00,0.00,0.00}{\textbf{{#1}}}}
    \newcommand{\FunctionTok}[1]{\textcolor[rgb]{0.02,0.16,0.49}{{#1}}}
    \newcommand{\RegionMarkerTok}[1]{{#1}}
    \newcommand{\ErrorTok}[1]{\textcolor[rgb]{1.00,0.00,0.00}{\textbf{{#1}}}}
    \newcommand{\NormalTok}[1]{{#1}}

    % Additional commands for more recent versions of Pandoc
    \newcommand{\ConstantTok}[1]{\textcolor[rgb]{0.53,0.00,0.00}{{#1}}}
    \newcommand{\SpecialCharTok}[1]{\textcolor[rgb]{0.25,0.44,0.63}{{#1}}}
    \newcommand{\VerbatimStringTok}[1]{\textcolor[rgb]{0.25,0.44,0.63}{{#1}}}
    \newcommand{\SpecialStringTok}[1]{\textcolor[rgb]{0.73,0.40,0.53}{{#1}}}
    \newcommand{\ImportTok}[1]{{#1}}
    \newcommand{\DocumentationTok}[1]{\textcolor[rgb]{0.73,0.13,0.13}{\textit{{#1}}}}
    \newcommand{\AnnotationTok}[1]{\textcolor[rgb]{0.38,0.63,0.69}{\textbf{\textit{{#1}}}}}
    \newcommand{\CommentVarTok}[1]{\textcolor[rgb]{0.38,0.63,0.69}{\textbf{\textit{{#1}}}}}
    \newcommand{\VariableTok}[1]{\textcolor[rgb]{0.10,0.09,0.49}{{#1}}}
    \newcommand{\ControlFlowTok}[1]{\textcolor[rgb]{0.00,0.44,0.13}{\textbf{{#1}}}}
    \newcommand{\OperatorTok}[1]{\textcolor[rgb]{0.40,0.40,0.40}{{#1}}}
    \newcommand{\BuiltInTok}[1]{{#1}}
    \newcommand{\ExtensionTok}[1]{{#1}}
    \newcommand{\PreprocessorTok}[1]{\textcolor[rgb]{0.74,0.48,0.00}{{#1}}}
    \newcommand{\AttributeTok}[1]{\textcolor[rgb]{0.49,0.56,0.16}{{#1}}}
    \newcommand{\InformationTok}[1]{\textcolor[rgb]{0.38,0.63,0.69}{\textbf{\textit{{#1}}}}}
    \newcommand{\WarningTok}[1]{\textcolor[rgb]{0.38,0.63,0.69}{\textbf{\textit{{#1}}}}}


    % Define a nice break command that doesn't care if a line doesn't already
    % exist.
    \def\br{\hspace*{\fill} \\* }
    % Math Jax compatibility definitions
    \def\gt{>}
    \def\lt{<}
    \let\Oldtex\TeX
    \let\Oldlatex\LaTeX
    \renewcommand{\TeX}{\textrm{\Oldtex}}
    \renewcommand{\LaTeX}{\textrm{\Oldlatex}}
    % Document parameters
    % Document title
    \title{Trabalho 2 - Regressão}
    \author{Carolina Santiago de Medeiros - DRE 122053305}
    \date{}
    
    
    
% Pygments definitions
\makeatletter
\def\PY@reset{\let\PY@it=\relax \let\PY@bf=\relax%
    \let\PY@ul=\relax \let\PY@tc=\relax%
    \let\PY@bc=\relax \let\PY@ff=\relax}
\def\PY@tok#1{\csname PY@tok@#1\endcsname}
\def\PY@toks#1+{\ifx\relax#1\empty\else%
    \PY@tok{#1}\expandafter\PY@toks\fi}
\def\PY@do#1{\PY@bc{\PY@tc{\PY@ul{%
    \PY@it{\PY@bf{\PY@ff{#1}}}}}}}
\def\PY#1#2{\PY@reset\PY@toks#1+\relax+\PY@do{#2}}

\@namedef{PY@tok@w}{\def\PY@tc##1{\textcolor[rgb]{0.73,0.73,0.73}{##1}}}
\@namedef{PY@tok@c}{\let\PY@it=\textit\def\PY@tc##1{\textcolor[rgb]{0.24,0.48,0.48}{##1}}}
\@namedef{PY@tok@cp}{\def\PY@tc##1{\textcolor[rgb]{0.61,0.40,0.00}{##1}}}
\@namedef{PY@tok@k}{\let\PY@bf=\textbf\def\PY@tc##1{\textcolor[rgb]{0.00,0.50,0.00}{##1}}}
\@namedef{PY@tok@kp}{\def\PY@tc##1{\textcolor[rgb]{0.00,0.50,0.00}{##1}}}
\@namedef{PY@tok@kt}{\def\PY@tc##1{\textcolor[rgb]{0.69,0.00,0.25}{##1}}}
\@namedef{PY@tok@o}{\def\PY@tc##1{\textcolor[rgb]{0.40,0.40,0.40}{##1}}}
\@namedef{PY@tok@ow}{\let\PY@bf=\textbf\def\PY@tc##1{\textcolor[rgb]{0.67,0.13,1.00}{##1}}}
\@namedef{PY@tok@nb}{\def\PY@tc##1{\textcolor[rgb]{0.00,0.50,0.00}{##1}}}
\@namedef{PY@tok@nf}{\def\PY@tc##1{\textcolor[rgb]{0.00,0.00,1.00}{##1}}}
\@namedef{PY@tok@nc}{\let\PY@bf=\textbf\def\PY@tc##1{\textcolor[rgb]{0.00,0.00,1.00}{##1}}}
\@namedef{PY@tok@nn}{\let\PY@bf=\textbf\def\PY@tc##1{\textcolor[rgb]{0.00,0.00,1.00}{##1}}}
\@namedef{PY@tok@ne}{\let\PY@bf=\textbf\def\PY@tc##1{\textcolor[rgb]{0.80,0.25,0.22}{##1}}}
\@namedef{PY@tok@nv}{\def\PY@tc##1{\textcolor[rgb]{0.10,0.09,0.49}{##1}}}
\@namedef{PY@tok@no}{\def\PY@tc##1{\textcolor[rgb]{0.53,0.00,0.00}{##1}}}
\@namedef{PY@tok@nl}{\def\PY@tc##1{\textcolor[rgb]{0.46,0.46,0.00}{##1}}}
\@namedef{PY@tok@ni}{\let\PY@bf=\textbf\def\PY@tc##1{\textcolor[rgb]{0.44,0.44,0.44}{##1}}}
\@namedef{PY@tok@na}{\def\PY@tc##1{\textcolor[rgb]{0.41,0.47,0.13}{##1}}}
\@namedef{PY@tok@nt}{\let\PY@bf=\textbf\def\PY@tc##1{\textcolor[rgb]{0.00,0.50,0.00}{##1}}}
\@namedef{PY@tok@nd}{\def\PY@tc##1{\textcolor[rgb]{0.67,0.13,1.00}{##1}}}
\@namedef{PY@tok@s}{\def\PY@tc##1{\textcolor[rgb]{0.73,0.13,0.13}{##1}}}
\@namedef{PY@tok@sd}{\let\PY@it=\textit\def\PY@tc##1{\textcolor[rgb]{0.73,0.13,0.13}{##1}}}
\@namedef{PY@tok@si}{\let\PY@bf=\textbf\def\PY@tc##1{\textcolor[rgb]{0.64,0.35,0.47}{##1}}}
\@namedef{PY@tok@se}{\let\PY@bf=\textbf\def\PY@tc##1{\textcolor[rgb]{0.67,0.36,0.12}{##1}}}
\@namedef{PY@tok@sr}{\def\PY@tc##1{\textcolor[rgb]{0.64,0.35,0.47}{##1}}}
\@namedef{PY@tok@ss}{\def\PY@tc##1{\textcolor[rgb]{0.10,0.09,0.49}{##1}}}
\@namedef{PY@tok@sx}{\def\PY@tc##1{\textcolor[rgb]{0.00,0.50,0.00}{##1}}}
\@namedef{PY@tok@m}{\def\PY@tc##1{\textcolor[rgb]{0.40,0.40,0.40}{##1}}}
\@namedef{PY@tok@gh}{\let\PY@bf=\textbf\def\PY@tc##1{\textcolor[rgb]{0.00,0.00,0.50}{##1}}}
\@namedef{PY@tok@gu}{\let\PY@bf=\textbf\def\PY@tc##1{\textcolor[rgb]{0.50,0.00,0.50}{##1}}}
\@namedef{PY@tok@gd}{\def\PY@tc##1{\textcolor[rgb]{0.63,0.00,0.00}{##1}}}
\@namedef{PY@tok@gi}{\def\PY@tc##1{\textcolor[rgb]{0.00,0.52,0.00}{##1}}}
\@namedef{PY@tok@gr}{\def\PY@tc##1{\textcolor[rgb]{0.89,0.00,0.00}{##1}}}
\@namedef{PY@tok@ge}{\let\PY@it=\textit}
\@namedef{PY@tok@gs}{\let\PY@bf=\textbf}
\@namedef{PY@tok@gp}{\let\PY@bf=\textbf\def\PY@tc##1{\textcolor[rgb]{0.00,0.00,0.50}{##1}}}
\@namedef{PY@tok@go}{\def\PY@tc##1{\textcolor[rgb]{0.44,0.44,0.44}{##1}}}
\@namedef{PY@tok@gt}{\def\PY@tc##1{\textcolor[rgb]{0.00,0.27,0.87}{##1}}}
\@namedef{PY@tok@err}{\def\PY@bc##1{{\setlength{\fboxsep}{\string -\fboxrule}\fcolorbox[rgb]{1.00,0.00,0.00}{1,1,1}{\strut ##1}}}}
\@namedef{PY@tok@kc}{\let\PY@bf=\textbf\def\PY@tc##1{\textcolor[rgb]{0.00,0.50,0.00}{##1}}}
\@namedef{PY@tok@kd}{\let\PY@bf=\textbf\def\PY@tc##1{\textcolor[rgb]{0.00,0.50,0.00}{##1}}}
\@namedef{PY@tok@kn}{\let\PY@bf=\textbf\def\PY@tc##1{\textcolor[rgb]{0.00,0.50,0.00}{##1}}}
\@namedef{PY@tok@kr}{\let\PY@bf=\textbf\def\PY@tc##1{\textcolor[rgb]{0.00,0.50,0.00}{##1}}}
\@namedef{PY@tok@bp}{\def\PY@tc##1{\textcolor[rgb]{0.00,0.50,0.00}{##1}}}
\@namedef{PY@tok@fm}{\def\PY@tc##1{\textcolor[rgb]{0.00,0.00,1.00}{##1}}}
\@namedef{PY@tok@vc}{\def\PY@tc##1{\textcolor[rgb]{0.10,0.09,0.49}{##1}}}
\@namedef{PY@tok@vg}{\def\PY@tc##1{\textcolor[rgb]{0.10,0.09,0.49}{##1}}}
\@namedef{PY@tok@vi}{\def\PY@tc##1{\textcolor[rgb]{0.10,0.09,0.49}{##1}}}
\@namedef{PY@tok@vm}{\def\PY@tc##1{\textcolor[rgb]{0.10,0.09,0.49}{##1}}}
\@namedef{PY@tok@sa}{\def\PY@tc##1{\textcolor[rgb]{0.73,0.13,0.13}{##1}}}
\@namedef{PY@tok@sb}{\def\PY@tc##1{\textcolor[rgb]{0.73,0.13,0.13}{##1}}}
\@namedef{PY@tok@sc}{\def\PY@tc##1{\textcolor[rgb]{0.73,0.13,0.13}{##1}}}
\@namedef{PY@tok@dl}{\def\PY@tc##1{\textcolor[rgb]{0.73,0.13,0.13}{##1}}}
\@namedef{PY@tok@s2}{\def\PY@tc##1{\textcolor[rgb]{0.73,0.13,0.13}{##1}}}
\@namedef{PY@tok@sh}{\def\PY@tc##1{\textcolor[rgb]{0.73,0.13,0.13}{##1}}}
\@namedef{PY@tok@s1}{\def\PY@tc##1{\textcolor[rgb]{0.73,0.13,0.13}{##1}}}
\@namedef{PY@tok@mb}{\def\PY@tc##1{\textcolor[rgb]{0.40,0.40,0.40}{##1}}}
\@namedef{PY@tok@mf}{\def\PY@tc##1{\textcolor[rgb]{0.40,0.40,0.40}{##1}}}
\@namedef{PY@tok@mh}{\def\PY@tc##1{\textcolor[rgb]{0.40,0.40,0.40}{##1}}}
\@namedef{PY@tok@mi}{\def\PY@tc##1{\textcolor[rgb]{0.40,0.40,0.40}{##1}}}
\@namedef{PY@tok@il}{\def\PY@tc##1{\textcolor[rgb]{0.40,0.40,0.40}{##1}}}
\@namedef{PY@tok@mo}{\def\PY@tc##1{\textcolor[rgb]{0.40,0.40,0.40}{##1}}}
\@namedef{PY@tok@ch}{\let\PY@it=\textit\def\PY@tc##1{\textcolor[rgb]{0.24,0.48,0.48}{##1}}}
\@namedef{PY@tok@cm}{\let\PY@it=\textit\def\PY@tc##1{\textcolor[rgb]{0.24,0.48,0.48}{##1}}}
\@namedef{PY@tok@cpf}{\let\PY@it=\textit\def\PY@tc##1{\textcolor[rgb]{0.24,0.48,0.48}{##1}}}
\@namedef{PY@tok@c1}{\let\PY@it=\textit\def\PY@tc##1{\textcolor[rgb]{0.24,0.48,0.48}{##1}}}
\@namedef{PY@tok@cs}{\let\PY@it=\textit\def\PY@tc##1{\textcolor[rgb]{0.24,0.48,0.48}{##1}}}

\def\PYZbs{\char`\\}
\def\PYZus{\char`\_}
\def\PYZob{\char`\{}
\def\PYZcb{\char`\}}
\def\PYZca{\char`\^}
\def\PYZam{\char`\&}
\def\PYZlt{\char`\<}
\def\PYZgt{\char`\>}
\def\PYZsh{\char`\#}
\def\PYZpc{\char`\%}
\def\PYZdl{\char`\$}
\def\PYZhy{\char`\-}
\def\PYZsq{\char`\'}
\def\PYZdq{\char`\"}
\def\PYZti{\char`\~}
% for compatibility with earlier versions
\def\PYZat{@}
\def\PYZlb{[}
\def\PYZrb{]}
\makeatother


    % For linebreaks inside Verbatim environment from package fancyvrb.
    \makeatletter
        \newbox\Wrappedcontinuationbox
        \newbox\Wrappedvisiblespacebox
        \newcommand*\Wrappedvisiblespace {\textcolor{red}{\textvisiblespace}}
        \newcommand*\Wrappedcontinuationsymbol {\textcolor{red}{\llap{\tiny$\m@th\hookrightarrow$}}}
        \newcommand*\Wrappedcontinuationindent {3ex }
        \newcommand*\Wrappedafterbreak {\kern\Wrappedcontinuationindent\copy\Wrappedcontinuationbox}
        % Take advantage of the already applied Pygments mark-up to insert
        % potential linebreaks for TeX processing.
        %        {, <, #, %, $, ' and ": go to next line.
        %        _, }, ^, &, >, - and ~: stay at end of broken line.
        % Use of \textquotesingle for straight quote.
        \newcommand*\Wrappedbreaksatspecials {%
            \def\PYGZus{\discretionary{\char`\_}{\Wrappedafterbreak}{\char`\_}}%
            \def\PYGZob{\discretionary{}{\Wrappedafterbreak\char`\{}{\char`\{}}%
            \def\PYGZcb{\discretionary{\char`\}}{\Wrappedafterbreak}{\char`\}}}%
            \def\PYGZca{\discretionary{\char`\^}{\Wrappedafterbreak}{\char`\^}}%
            \def\PYGZam{\discretionary{\char`\&}{\Wrappedafterbreak}{\char`\&}}%
            \def\PYGZlt{\discretionary{}{\Wrappedafterbreak\char`\<}{\char`\<}}%
            \def\PYGZgt{\discretionary{\char`\>}{\Wrappedafterbreak}{\char`\>}}%
            \def\PYGZsh{\discretionary{}{\Wrappedafterbreak\char`\#}{\char`\#}}%
            \def\PYGZpc{\discretionary{}{\Wrappedafterbreak\char`\%}{\char`\%}}%
            \def\PYGZdl{\discretionary{}{\Wrappedafterbreak\char`\$}{\char`\$}}%
            \def\PYGZhy{\discretionary{\char`\-}{\Wrappedafterbreak}{\char`\-}}%
            \def\PYGZsq{\discretionary{}{\Wrappedafterbreak\textquotesingle}{\textquotesingle}}%
            \def\PYGZdq{\discretionary{}{\Wrappedafterbreak\char`\"}{\char`\"}}%
            \def\PYGZti{\discretionary{\char`\~}{\Wrappedafterbreak}{\char`\~}}%
        }
        % Some characters . , ; ? ! / are not pygmentized.
        % This macro makes them "active" and they will insert potential linebreaks
        \newcommand*\Wrappedbreaksatpunct {%
            \lccode`\~`\.\lowercase{\def~}{\discretionary{\hbox{\char`\.}}{\Wrappedafterbreak}{\hbox{\char`\.}}}%
            \lccode`\~`\,\lowercase{\def~}{\discretionary{\hbox{\char`\,}}{\Wrappedafterbreak}{\hbox{\char`\,}}}%
            \lccode`\~`\;\lowercase{\def~}{\discretionary{\hbox{\char`\;}}{\Wrappedafterbreak}{\hbox{\char`\;}}}%
            \lccode`\~`\:\lowercase{\def~}{\discretionary{\hbox{\char`\:}}{\Wrappedafterbreak}{\hbox{\char`\:}}}%
            \lccode`\~`\?\lowercase{\def~}{\discretionary{\hbox{\char`\?}}{\Wrappedafterbreak}{\hbox{\char`\?}}}%
            \lccode`\~`\!\lowercase{\def~}{\discretionary{\hbox{\char`\!}}{\Wrappedafterbreak}{\hbox{\char`\!}}}%
            \lccode`\~`\/\lowercase{\def~}{\discretionary{\hbox{\char`\/}}{\Wrappedafterbreak}{\hbox{\char`\/}}}%
            \catcode`\.\active
            \catcode`\,\active
            \catcode`\;\active
            \catcode`\:\active
            \catcode`\?\active
            \catcode`\!\active
            \catcode`\/\active
            \lccode`\~`\~
        }
    \makeatother

    \let\OriginalVerbatim=\Verbatim
    \makeatletter
    \renewcommand{\Verbatim}[1][1]{%
        %\parskip\z@skip
        \sbox\Wrappedcontinuationbox {\Wrappedcontinuationsymbol}%
        \sbox\Wrappedvisiblespacebox {\FV@SetupFont\Wrappedvisiblespace}%
        \def\FancyVerbFormatLine ##1{\hsize\linewidth
            \vtop{\raggedright\hyphenpenalty\z@\exhyphenpenalty\z@
                \doublehyphendemerits\z@\finalhyphendemerits\z@
                \strut ##1\strut}%
        }%
        % If the linebreak is at a space, the latter will be displayed as visible
        % space at end of first line, and a continuation symbol starts next line.
        % Stretch/shrink are however usually zero for typewriter font.
        \def\FV@Space {%
            \nobreak\hskip\z@ plus\fontdimen3\font minus\fontdimen4\font
            \discretionary{\copy\Wrappedvisiblespacebox}{\Wrappedafterbreak}
            {\kern\fontdimen2\font}%
        }%

        % Allow breaks at special characters using \PYG... macros.
        \Wrappedbreaksatspecials
        % Breaks at punctuation characters . , ; ? ! and / need catcode=\active
        \OriginalVerbatim[#1,codes*=\Wrappedbreaksatpunct]%
    }
    \makeatother

    % Exact colors from NB
    \definecolor{incolor}{HTML}{303F9F}
    \definecolor{outcolor}{HTML}{D84315}
    \definecolor{cellborder}{HTML}{CFCFCF}
    \definecolor{cellbackground}{HTML}{F7F7F7}

    % prompt
    \makeatletter
    \newcommand{\boxspacing}{\kern\kvtcb@left@rule\kern\kvtcb@boxsep}
    \makeatother
    \newcommand{\prompt}[4]{
        {\ttfamily\llap{{\color{#2}[#3]:\hspace{3pt}#4}}\vspace{-\baselineskip}}
    }
    

    
    % Prevent overflowing lines due to hard-to-break entities
    \sloppy
    % Setup hyperref package
    \hypersetup{
      breaklinks=true,  % so long urls are correctly broken across lines
      colorlinks=true,
      urlcolor=urlcolor,
      linkcolor=linkcolor,
      citecolor=citecolor,
      }
    % Slightly bigger margins than the latex defaults
    
    \geometry{verbose,tmargin=1in,bmargin=1in,lmargin=1in,rmargin=1in}
    
    

\begin{document}
    
    \maketitle
    
    

    
    \hypertarget{cuxf3digo-trabalho-2-regressuxe3o}{%
\section{Código Trabalho 2
(Regressão)}\label{cuxf3digo-trabalho-2-regressuxe3o}}

    \hypertarget{definindo-os-dataframes-de-treinamento-e-de-teste}{%
\subsection{Definindo os DataFrames de treinamento e de
teste}\label{definindo-os-dataframes-de-treinamento-e-de-teste}}

    \begin{tcolorbox}[breakable, size=fbox, boxrule=1pt, pad at break*=1mm,colback=cellbackground, colframe=cellborder]
\prompt{In}{incolor}{ }{\boxspacing}
\begin{Verbatim}[commandchars=\\\{\}]
\PY{k+kn}{import} \PY{n+nn}{pandas} \PY{k}{as} \PY{n+nn}{pd}

\PY{n}{df\PYZus{}train} \PY{o}{=} \PY{n}{pd}\PY{o}{.}\PY{n}{read\PYZus{}csv}\PY{p}{(}\PY{l+s+s1}{\PYZsq{}}\PY{l+s+s1}{Dataset/conjunto\PYZus{}de\PYZus{}treinamento.csv}\PY{l+s+s1}{\PYZsq{}}\PY{p}{)}
\PY{n}{df\PYZus{}test} \PY{o}{=} \PY{n}{pd}\PY{o}{.}\PY{n}{read\PYZus{}csv}\PY{p}{(}\PY{l+s+s1}{\PYZsq{}}\PY{l+s+s1}{Dataset/conjunto\PYZus{}de\PYZus{}teste.csv}\PY{l+s+s1}{\PYZsq{}}\PY{p}{)}

\PY{n}{df\PYZus{}train}\PY{o}{.}\PY{n}{shape}
\end{Verbatim}
\end{tcolorbox}

    \hypertarget{pruxe9-processamento-dos-dados}{%
\subsection{Pré-processamento dos
dados}\label{pruxe9-processamento-dos-dados}}

    \hypertarget{funuxe7uxf5es-usadas-para-tratar-os-dados}{%
\subsubsection{Funções usadas para ``tratar'' os
dados:}\label{funuxe7uxf5es-usadas-para-tratar-os-dados}}

\begin{itemize}
\tightlist
\item
  Atribuir valores numéricos às variáveis nominais;
\item
  Remover variáveis que não trazem boas contribuições para o resultado
  final (olhar a sessão de ``análise de dados'' ao final do código)
\item
  Preenche espaços em branco (\texttt{null}) nas colunas que precisam
\end{itemize}

    \begin{tcolorbox}[breakable, size=fbox, boxrule=1pt, pad at break*=1mm,colback=cellbackground, colframe=cellborder]
\prompt{In}{incolor}{ }{\boxspacing}
\begin{Verbatim}[commandchars=\\\{\}]
\PY{k}{def} \PY{n+nf}{DataProcessing1}\PY{p}{(}\PY{n}{df}\PY{p}{)}\PY{p}{:}
    \PY{c+c1}{\PYZsh{} Exclui colunas que não serão utilizadas}
    \PY{n}{df}\PY{o}{.}\PY{n}{drop}\PY{p}{(}\PY{n}{columns}\PY{o}{=}\PY{p}{[}\PY{l+s+s1}{\PYZsq{}}\PY{l+s+s1}{Id}\PY{l+s+s1}{\PYZsq{}}\PY{p}{,} \PY{l+s+s1}{\PYZsq{}}\PY{l+s+s1}{diferenciais}\PY{l+s+s1}{\PYZsq{}}\PY{p}{,} \PY{l+s+s1}{\PYZsq{}}\PY{l+s+s1}{tipo\PYZus{}vendedor}\PY{l+s+s1}{\PYZsq{}}\PY{p}{,} \PY{l+s+s1}{\PYZsq{}}\PY{l+s+s1}{area\PYZus{}extra}\PY{l+s+s1}{\PYZsq{}}\PY{p}{,} \PY{l+s+s1}{\PYZsq{}}\PY{l+s+s1}{estacionamento}\PY{l+s+s1}{\PYZsq{}}\PY{p}{]}\PY{p}{,} \PY{n}{inplace}\PY{o}{=}\PY{k+kc}{True}\PY{p}{)}
    \PY{n}{df}\PY{o}{.}\PY{n}{drop}\PY{p}{(}\PY{n}{columns}\PY{o}{=}\PY{p}{[}\PY{l+s+s1}{\PYZsq{}}\PY{l+s+s1}{quartos}\PY{l+s+s1}{\PYZsq{}}\PY{p}{,} \PY{l+s+s1}{\PYZsq{}}\PY{l+s+s1}{piscina}\PY{l+s+s1}{\PYZsq{}}\PY{p}{,} \PY{l+s+s1}{\PYZsq{}}\PY{l+s+s1}{s\PYZus{}ginastica}\PY{l+s+s1}{\PYZsq{}}\PY{p}{]}\PY{p}{,} \PY{n}{inplace}\PY{o}{=}\PY{k+kc}{True}\PY{p}{)}

    \PY{c+c1}{\PYZsh{} Transforma coluna \PYZsq{}tipo\PYZsq{} em numérica}
    \PY{n}{df}\PY{p}{[}\PY{l+s+s1}{\PYZsq{}}\PY{l+s+s1}{tipo}\PY{l+s+s1}{\PYZsq{}}\PY{p}{]} \PY{o}{=} \PY{n}{df}\PY{p}{[}\PY{l+s+s1}{\PYZsq{}}\PY{l+s+s1}{tipo}\PY{l+s+s1}{\PYZsq{}}\PY{p}{]}\PY{o}{.}\PY{n}{map}\PY{p}{(}\PY{p}{\PYZob{}}\PY{l+s+s1}{\PYZsq{}}\PY{l+s+s1}{Apartamento}\PY{l+s+s1}{\PYZsq{}}\PY{p}{:} \PY{l+m+mi}{1}\PY{p}{,} \PY{l+s+s1}{\PYZsq{}}\PY{l+s+s1}{Casa}\PY{l+s+s1}{\PYZsq{}}\PY{p}{:} \PY{l+m+mi}{2}\PY{p}{\PYZcb{}}\PY{p}{)}
    \PY{n}{df}\PY{o}{.}\PY{n}{fillna}\PY{p}{(}\PY{l+m+mf}{0.5}\PY{p}{,} \PY{n}{inplace}\PY{o}{=}\PY{k+kc}{True}\PY{p}{)}

    \PY{k}{return} \PY{n}{df}
\end{Verbatim}
\end{tcolorbox}

    Na função \texttt{DataProcessing1}, as variáveis
\texttt{\textquotesingle{}tipo\_vendedor\textquotesingle{}},
\texttt{\textquotesingle{}area\_extra\textquotesingle{}},
\texttt{\textquotesingle{}estacionamento\textquotesingle{}} foram
removidas por ter uma correlação muito baixa com as outras variáveis. Os
gráficos a seguir mostram a correlação entre as variáveis para cada uma
destas:

\begin{figure}
\centering
\includegraphics{C:/Users/carol/OneDrive/Documentos/GitHub/Introduction-to-Machine-Learning-EEL891/Trabalho 2 - Regressão/graficos correlação (variáveis excuídas)/tipo_vendedor.png}
\caption{vendedor}
\end{figure}

\begin{figure}
\centering
\includegraphics{C:/Users/carol/OneDrive/Documentos/GitHub/Introduction-to-Machine-Learning-EEL891/Trabalho 2 - Regressão/graficos correlação (variáveis excuídas)/area_extra.png}
\caption{areaextra}
\end{figure}

\begin{figure}
\centering
\includegraphics{C:/Users/carol/OneDrive/Documentos/GitHub/Introduction-to-Machine-Learning-EEL891/Trabalho 2 - Regressão/graficos correlação (variáveis excuídas)/estacionamento.png}
\caption{estacionamento}
\end{figure}

    \begin{tcolorbox}[breakable, size=fbox, boxrule=1pt, pad at break*=1mm,colback=cellbackground, colframe=cellborder]
\prompt{In}{incolor}{ }{\boxspacing}
\begin{Verbatim}[commandchars=\\\{\}]
\PY{k}{def} \PY{n+nf}{DataProcessing2}\PY{p}{(}\PY{n}{df}\PY{p}{)}\PY{p}{:}
    \PY{c+c1}{\PYZsh{} Calcula a média dos valores dos imóveis por bairro}
    \PY{n}{mean\PYZus{}encoded} \PY{o}{=} \PY{n}{df}\PY{o}{.}\PY{n}{groupby}\PY{p}{(}\PY{l+s+s1}{\PYZsq{}}\PY{l+s+s1}{bairro}\PY{l+s+s1}{\PYZsq{}}\PY{p}{)}\PY{p}{[}\PY{l+s+s1}{\PYZsq{}}\PY{l+s+s1}{preco}\PY{l+s+s1}{\PYZsq{}}\PY{p}{]}\PY{o}{.}\PY{n}{mean}\PY{p}{(}\PY{p}{)}
    
    \PY{c+c1}{\PYZsh{} Cria um dicionário com os bairros originais e suas respectivas versões codificadas (valores médios dos imóveis)}
    \PY{n}{bairros\PYZus{}config} \PY{o}{=} \PY{n}{mean\PYZus{}encoded}\PY{o}{.}\PY{n}{to\PYZus{}dict}\PY{p}{(}\PY{p}{)}
    
    \PY{c+c1}{\PYZsh{} Mapeia os valores médios dos imóveis por bairro para todos os bairros}
    \PY{n}{df}\PY{p}{[}\PY{l+s+s1}{\PYZsq{}}\PY{l+s+s1}{bairro}\PY{l+s+s1}{\PYZsq{}}\PY{p}{]} \PY{o}{=} \PY{n}{df}\PY{p}{[}\PY{l+s+s1}{\PYZsq{}}\PY{l+s+s1}{bairro}\PY{l+s+s1}{\PYZsq{}}\PY{p}{]}\PY{o}{.}\PY{n}{map}\PY{p}{(}\PY{n}{mean\PYZus{}encoded}\PY{p}{)}

    \PY{k}{return} \PY{n}{df}\PY{p}{,} \PY{n}{bairros\PYZus{}config}
\end{Verbatim}
\end{tcolorbox}

    \begin{tcolorbox}[breakable, size=fbox, boxrule=1pt, pad at break*=1mm,colback=cellbackground, colframe=cellborder]
\prompt{In}{incolor}{ }{\boxspacing}
\begin{Verbatim}[commandchars=\\\{\}]
\PY{k}{def} \PY{n+nf}{CodificarBairroTeste}\PY{p}{(}\PY{n}{df}\PY{p}{,} \PY{n}{bairros\PYZus{}config}\PY{p}{)}\PY{p}{:}
    \PY{c+c1}{\PYZsh{} Codifica a coluna \PYZsq{}bairro\PYZsq{} do conjunto de teste de acordo com o dicionário criado por DataProcessing2(df\PYZus{}train)}
    \PY{n}{df}\PY{p}{[}\PY{l+s+s1}{\PYZsq{}}\PY{l+s+s1}{bairro}\PY{l+s+s1}{\PYZsq{}}\PY{p}{]} \PY{o}{=} \PY{n}{df}\PY{p}{[}\PY{l+s+s1}{\PYZsq{}}\PY{l+s+s1}{bairro}\PY{l+s+s1}{\PYZsq{}}\PY{p}{]}\PY{o}{.}\PY{n}{map}\PY{p}{(}\PY{n}{bairros\PYZus{}config}\PY{p}{)}
    \PY{n}{df\PYZus{}test}\PY{p}{[}\PY{l+s+s1}{\PYZsq{}}\PY{l+s+s1}{bairro}\PY{l+s+s1}{\PYZsq{}}\PY{p}{]}\PY{o}{.}\PY{n}{fillna}\PY{p}{(}\PY{n}{df\PYZus{}test}\PY{p}{[}\PY{l+s+s1}{\PYZsq{}}\PY{l+s+s1}{bairro}\PY{l+s+s1}{\PYZsq{}}\PY{p}{]}\PY{o}{.}\PY{n}{mean}\PY{p}{(}\PY{p}{)}\PY{p}{,} \PY{n}{inplace}\PY{o}{=}\PY{k+kc}{True}\PY{p}{)}

    \PY{k}{return} \PY{n}{df}
\end{Verbatim}
\end{tcolorbox}

    \begin{tcolorbox}[breakable, size=fbox, boxrule=1pt, pad at break*=1mm,colback=cellbackground, colframe=cellborder]
\prompt{In}{incolor}{ }{\boxspacing}
\begin{Verbatim}[commandchars=\\\{\}]
\PY{c+c1}{\PYZsh{} Passa os DataFrames pelas funções de pré\PYZhy{}processamento}
\PY{n}{df\PYZus{}train} \PY{o}{=} \PY{n}{DataProcessing1}\PY{p}{(}\PY{n}{df\PYZus{}train}\PY{p}{)}
\PY{n}{df\PYZus{}test} \PY{o}{=} \PY{n}{DataProcessing1}\PY{p}{(}\PY{n}{df\PYZus{}test}\PY{p}{)}

\PY{n}{df\PYZus{}train}\PY{p}{,} \PY{n}{bairros\PYZus{}enc} \PY{o}{=} \PY{n}{DataProcessing2}\PY{p}{(}\PY{n}{df\PYZus{}train}\PY{p}{)}
\PY{n}{df\PYZus{}test} \PY{o}{=} \PY{n}{CodificarBairroTeste}\PY{p}{(}\PY{n}{df\PYZus{}test}\PY{p}{,} \PY{n}{bairros\PYZus{}enc}\PY{p}{)}
\end{Verbatim}
\end{tcolorbox}

    \hypertarget{checando-o-pruxe9-processamento}{%
\subsubsection{Checando o
pré-processamento}\label{checando-o-pruxe9-processamento}}

Verifica se todas as colunas dos DataFrames de treinamento e teste estão
completamente preenchidas, e se o formato dos dados é numérico.

    \begin{tcolorbox}[breakable, size=fbox, boxrule=1pt, pad at break*=1mm,colback=cellbackground, colframe=cellborder]
\prompt{In}{incolor}{ }{\boxspacing}
\begin{Verbatim}[commandchars=\\\{\}]
\PY{c+c1}{\PYZsh{} Conta o número de valores NaN em cada coluna}
\PY{n}{contagem\PYZus{}nan} \PY{o}{=} \PY{n}{df\PYZus{}test}\PY{o}{.}\PY{n}{isnull}\PY{p}{(}\PY{p}{)}\PY{o}{.}\PY{n}{sum}\PY{p}{(}\PY{p}{)}\PY{p}{,} \PY{n}{df\PYZus{}train}\PY{o}{.}\PY{n}{isnull}\PY{p}{(}\PY{p}{)}\PY{o}{.}\PY{n}{sum}\PY{p}{(}\PY{p}{)}
\PY{c+c1}{\PYZsh{} Exibe a contagem de valores NaN}
\PY{n+nb}{print}\PY{p}{(}\PY{n}{contagem\PYZus{}nan}\PY{p}{[}\PY{l+m+mi}{0}\PY{p}{]}\PY{p}{,} \PY{l+s+s1}{\PYZsq{}}\PY{l+s+se}{\PYZbs{}n}\PY{l+s+se}{\PYZbs{}n}\PY{l+s+s1}{\PYZsq{}}\PY{p}{,} \PY{n}{contagem\PYZus{}nan}\PY{p}{[}\PY{l+m+mi}{1}\PY{p}{]}\PY{p}{)}
\end{Verbatim}
\end{tcolorbox}

    \hypertarget{output-desejado}{%
\paragraph{Output desejado:}\label{output-desejado}}

\textbf{df\_test}
\begin{itemize}
    \item tipo 0
    \item bairro 0
    \item suites 0
    \item vagas 0
    \item area\_util 0
    \item churrasqueira 0
    \item playground 0
    \item quadra 0
    \item s\_festas 0
    \item s\_jogos 0
    \item sauna 0
    \item vista\_mar 0
\end{itemize}

\textbf{df\_train}
\begin{itemize}
    \item tipo 0
    \item bairro 0
    \item suites 0
    \item vagas 0
    \item area\_util 0
    \item churrasqueira 0
    \item playground 0
    \item quadra 0
    \item s\_festas 0
    \item s\_jogos 0
    \item sauna 0
    \item vista\_mar 0
    \item preco 0
\end{itemize}


    \hypertarget{treinamento-do-modelo}{%
\subsection{Treinamento do Modelo}\label{treinamento-do-modelo}}

O \texttt{HistGradientBoostingRegressor} é um modelo de regressão
baseado no algoritmo Gradient Boosting. A ideia principal por trás do
Gradient Boosting é combinar vários estimadores fracos (geralmente
árvores de decisão rasas) para formar um modelo mais poderoso. O
\texttt{HistGradientBoostingRegressor}, especificamente, é uma
implementação otimizada do Gradient Boosting que utiliza um algoritmo de
aprendizado de histograma para melhorar a eficiência e o desempenho.

O algoritmo de aprendizado de histograma divide os dados em intervalos
discretos (histogramas) e opera diretamente nesses histogramas, em vez
de usar valores individuais, tornando o processo de treinamento mais
rápido e reduzindo a utilização de memória.

Esse modelo é muito eficaz para problemas de regressão e é útil quando
se deseja prever um valor numérico a partir de um conjunto de
características.

\hypertarget{preparauxe7uxe3o-dos-dados}{%
\subsubsection{Preparação dos Dados}\label{preparauxe7uxe3o-dos-dados}}

Antes de treinar o modelo, é necessário preparar os dados de treinamento
e teste. Para isso, separamos as variáveis independentes
(\texttt{X\_train} e \texttt{X\_test}) e a variável alvo
(\texttt{y\_train}) do conjunto de treinamento.

\hypertarget{criauxe7uxe3o-e-treinamento-do-modelo}{%
\subsubsection{Criação e Treinamento do
Modelo}\label{criauxe7uxe3o-e-treinamento-do-modelo}}

Após a preparação dos dados, é criada uma instância do regressor
\texttt{HistGradientBoostingRegressor} com os parâmetros desejados para
configurar o modelo. Neste exemplo, serão utilizados alguns parâmetros
específicos:

\begin{itemize}
\item
  \texttt{l2\_regularization}: Um parâmetro de regularização L2 que
  controla o nível de regularização aplicado às funções de base. Valores
  maiores deste parâmetro aplicam mais regularização, ajudando a evitar
  overfitting (sobreajuste).
\item
  \texttt{max\_iter}: O número máximo de iterações (etapas) do boosting,
  ou seja, o número máximo de árvores de decisão que serão construídas
  durante o processo de treinamento.
\item
  \texttt{loss}: A função de perda utilizada para medir a qualidade das
  previsões. Neste caso, utilizamos ``absolute\_error'', que é a função
  de erro absoluto médio.
\item
  \texttt{max\_depth}: A profundidade máxima das árvores de decisão
  usadas no modelo.
\end{itemize}

    \begin{tcolorbox}[breakable, size=fbox, boxrule=1pt, pad at break*=1mm,colback=cellbackground, colframe=cellborder]
\prompt{In}{incolor}{ }{\boxspacing}
\begin{Verbatim}[commandchars=\\\{\}]
\PY{k+kn}{from} \PY{n+nn}{sklearn}\PY{n+nn}{.}\PY{n+nn}{ensemble} \PY{k+kn}{import} \PY{n}{HistGradientBoostingRegressor}

\PY{c+c1}{\PYZsh{} Separa os dados de treino e teste}
\PY{n}{X\PYZus{}train} \PY{o}{=} \PY{n}{df\PYZus{}train}\PY{o}{.}\PY{n}{iloc}\PY{p}{[}\PY{p}{:}\PY{p}{,} \PY{p}{:}\PY{o}{\PYZhy{}}\PY{l+m+mi}{1}\PY{p}{]}\PY{o}{.}\PY{n}{values}
\PY{n}{y\PYZus{}train} \PY{o}{=} \PY{n}{df\PYZus{}train}\PY{o}{.}\PY{n}{iloc}\PY{p}{[}\PY{p}{:}\PY{p}{,} \PY{o}{\PYZhy{}}\PY{l+m+mi}{1}\PY{p}{]}\PY{o}{.}\PY{n}{values}
\PY{n}{X\PYZus{}test} \PY{o}{=} \PY{n}{df\PYZus{}test}\PY{o}{.}\PY{n}{iloc}\PY{p}{[}\PY{p}{:}\PY{p}{,} \PY{p}{:}\PY{p}{]}\PY{o}{.}\PY{n}{values}

\PY{n}{X\PYZus{}train}\PY{o}{.}\PY{n}{shape}\PY{p}{,} \PY{n}{X\PYZus{}test}\PY{o}{.}\PY{n}{shape}

\PY{c+c1}{\PYZsh{} Cria e treina o modelo}
\PY{n}{model} \PY{o}{=} \PY{n}{HistGradientBoostingRegressor}\PY{p}{(}\PY{n}{l2\PYZus{}regularization}\PY{o}{=}\PY{l+m+mi}{34}\PY{p}{,} \PY{n}{max\PYZus{}iter}\PY{o}{=}\PY{l+m+mi}{140}\PY{p}{,} \PY{n}{loss} \PY{o}{=} \PY{l+s+s2}{\PYZdq{}}\PY{l+s+s2}{absolute\PYZus{}error}\PY{l+s+s2}{\PYZdq{}}\PY{p}{,} \PY{n}{max\PYZus{}depth}\PY{o}{=}\PY{l+m+mi}{12}\PY{p}{)}
\PY{n}{model}\PY{o}{.}\PY{n}{fit}\PY{p}{(}\PY{n}{X\PYZus{}train}\PY{p}{,} \PY{n}{y\PYZus{}train}\PY{p}{)}
\PY{n}{y\PYZus{}pred} \PY{o}{=} \PY{n}{model}\PY{o}{.}\PY{n}{predict}\PY{p}{(}\PY{n}{X\PYZus{}test}\PY{p}{)}
\end{Verbatim}
\end{tcolorbox}

    \hypertarget{salvando-resultados}{%
\subsubsection{Salvando resultados}\label{salvando-resultados}}

    \begin{tcolorbox}[breakable, size=fbox, boxrule=1pt, pad at break*=1mm,colback=cellbackground, colframe=cellborder]
\prompt{In}{incolor}{ }{\boxspacing}
\begin{Verbatim}[commandchars=\\\{\}]
\PY{c+c1}{\PYZsh{} Cria um DataFrame com a coluna \PYZsq{}Id\PYZsq{} e \PYZsq{}preco\PYZsq{} contendo os valores de y\PYZus{}pred}
\PY{n}{df\PYZus{}resultado} \PY{o}{=} \PY{n}{pd}\PY{o}{.}\PY{n}{DataFrame}\PY{p}{(}\PY{p}{\PYZob{}}\PY{l+s+s1}{\PYZsq{}}\PY{l+s+s1}{Id}\PY{l+s+s1}{\PYZsq{}}\PY{p}{:} \PY{n+nb}{range}\PY{p}{(}\PY{n+nb}{len}\PY{p}{(}\PY{n}{y\PYZus{}pred}\PY{p}{)}\PY{p}{)}\PY{p}{,} \PY{l+s+s1}{\PYZsq{}}\PY{l+s+s1}{preco}\PY{l+s+s1}{\PYZsq{}}\PY{p}{:} \PY{n}{y\PYZus{}pred}\PY{p}{\PYZcb{}}\PY{p}{)}

\PY{c+c1}{\PYZsh{} Salva o DataFrame em um arquivo CSV}
\PY{n}{df\PYZus{}resultado}\PY{o}{.}\PY{n}{to\PYZus{}csv}\PY{p}{(}\PY{l+s+s1}{\PYZsq{}}\PY{l+s+s1}{resultadofinalhist.csv}\PY{l+s+s1}{\PYZsq{}}\PY{p}{,} \PY{n}{index}\PY{o}{=}\PY{k+kc}{False}\PY{p}{)}
\PY{n}{df\PYZus{}resultado}\PY{o}{.}\PY{n}{shape}
\end{Verbatim}
\end{tcolorbox}

    \hypertarget{anuxe1lise-de-dados}{%
\subsection{Análise de dados}\label{anuxe1lise-de-dados}}

\hypertarget{funuxe7uxe3o-plot_feature_importance}{%
\subsubsection{Função
plot\_feature\_importance}\label{funuxe7uxe3o-plot_feature_importance}}

A função \texttt{plot\_feature\_importance} tem o objetivo de calcular e
visualizar a importância das características (recursos) do modelo
\texttt{HistGradientBoostingRegressor}. Essa importância é uma medida
que indica o impacto que cada característica tem no desempenho do modelo
para fazer previsões.

O cálculo da importância das características é feito usando a técnica de
importância de permutação (\texttt{permutation\_importance}). Essa
técnica consiste em medir o efeito de embaralhar aleatoriamente os
valores de uma característica e verificar como isso afeta o desempenho
do modelo. Se a importância de uma característica for alta, significa
que embaralhar seus valores causará uma redução significativa na
precisão do modelo, indicando que a característica é importante para
fazer previsões precisas.

A função \texttt{plot\_feature\_importance} recebe os seguintes
parâmetros:

\begin{itemize}
\tightlist
\item
  \texttt{model}: O modelo treinado
  \texttt{HistGradientBoostingRegressor}.
\item
  \texttt{X}: A matriz de características (recursos) de treinamento.
\item
  \texttt{y}: O vetor de valores alvo (rótulos) correspondentes às
  amostras em \texttt{X}.
\item
  \texttt{feature\_names}: Uma lista com os nomes das características.
\item
  A função utiliza a biblioteca permutation\_importance do scikit-learn
  para calcular as importâncias das características;
\item
  Em seguida, ela organiza as características em ordem decrescente de
  importância e plota um gráfico de barras para visualizar a importância
  de cada característica.
\end{itemize}

Além disso, a função também imprime uma lista
\texttt{unimportant\_features} que, como o nome sugere, contém as
características que têm importância igual a zero, o que indica que essas
características provavelmente não estão contribuindo para o desempenho
do modelo e poderiam ser consideradas para remoção.

Essa função , em específico sua propriedade de listar as features menos
importantes, foi utilizada para identificar quais variáveis deveriam ser
removidas. A partir da análise do gráfico de barras, foi possível
identificar que as variáveis
\texttt{\textquotesingle{}quartos\textquotesingle{}},
\texttt{\textquotesingle{}piscina\textquotesingle{}} e
\texttt{\textquotesingle{}s\_ginastica\textquotesingle{}} não contribuem
para o desempenho do modelo e, por isso, foram removidas. A seguir, o
gráfico de barras com a importância das características do modelo antes
da remoção dessas variáveis:

\begin{figure}
\centering
\includegraphics{C:/Users/carol/OneDrive/Documentos/GitHub/Introduction-to-Machine-Learning-EEL891/Trabalho 2 - Regressão/graficos correlação (variáveis excuídas)/preremocaoquartos.png}
\caption{Gráfico pré-remoção}
\end{figure}

Pós a remoção das variávies, o gráfico de barras ficou da seguinte
forma:

\begin{figure}
\centering
\includegraphics{C:/Users/carol/OneDrive/Documentos/GitHub/Introduction-to-Machine-Learning-EEL891/Trabalho 2 - Regressão/graficos correlação (variáveis excuídas)/posremocao.png}
\caption{Gráfico pré-remoção}
\end{figure}

A escala não permite tal percepção, porém, as variáveis
\texttt{\textquotesingle{}quadra\textquotesingle{}} e
\texttt{\textquotesingle{}s\_jogos\textquotesingle{}} não têm
importância nula segundo a função.

    \begin{tcolorbox}[breakable, size=fbox, boxrule=1pt, pad at break*=1mm,colback=cellbackground, colframe=cellborder]
\prompt{In}{incolor}{ }{\boxspacing}
\begin{Verbatim}[commandchars=\\\{\}]
\PY{k+kn}{import} \PY{n+nn}{matplotlib}\PY{n+nn}{.}\PY{n+nn}{pyplot} \PY{k}{as} \PY{n+nn}{plt}
\PY{k+kn}{import} \PY{n+nn}{seaborn} \PY{k}{as} \PY{n+nn}{sns}
\PY{k+kn}{from} \PY{n+nn}{sklearn}\PY{n+nn}{.}\PY{n+nn}{ensemble} \PY{k+kn}{import} \PY{n}{HistGradientBoostingRegressor}
\PY{k+kn}{from} \PY{n+nn}{sklearn}\PY{n+nn}{.}\PY{n+nn}{inspection} \PY{k+kn}{import} \PY{n}{permutation\PYZus{}importance}

\PY{k}{def} \PY{n+nf}{plot\PYZus{}feature\PYZus{}importance}\PY{p}{(}\PY{n}{model}\PY{p}{,} \PY{n}{X}\PY{p}{,} \PY{n}{y}\PY{p}{,} \PY{n}{feature\PYZus{}names}\PY{p}{)}\PY{p}{:}
    \PY{c+c1}{\PYZsh{} Define metodo de calcular importancias}
    \PY{n}{result} \PY{o}{=} \PY{n}{permutation\PYZus{}importance}\PY{p}{(}\PY{n}{model}\PY{p}{,} \PY{n}{X}\PY{p}{,} \PY{n}{y}\PY{p}{,} \PY{n}{n\PYZus{}repeats}\PY{o}{=}\PY{l+m+mi}{10}\PY{p}{,} \PY{n}{random\PYZus{}state}\PY{o}{=}\PY{l+m+mi}{42}\PY{p}{)}

    \PY{c+c1}{\PYZsh{} Calcula as importancias e suas incertezas}
    \PY{n}{feature\PYZus{}importance} \PY{o}{=} \PY{n}{result}\PY{o}{.}\PY{n}{importances\PYZus{}mean}
    \PY{n}{feature\PYZus{}importance\PYZus{}std} \PY{o}{=} \PY{n}{result}\PY{o}{.}\PY{n}{importances\PYZus{}std}

    \PY{c+c1}{\PYZsh{} Cria um df com importancias e incertezas}
    \PY{n}{feature\PYZus{}importance\PYZus{}df} \PY{o}{=} \PY{n}{pd}\PY{o}{.}\PY{n}{DataFrame}\PY{p}{(}\PY{p}{\PYZob{}}\PY{l+s+s1}{\PYZsq{}}\PY{l+s+s1}{Feature}\PY{l+s+s1}{\PYZsq{}}\PY{p}{:} \PY{n}{feature\PYZus{}names}\PY{p}{,} \PY{l+s+s1}{\PYZsq{}}\PY{l+s+s1}{Importance}\PY{l+s+s1}{\PYZsq{}}\PY{p}{:} \PY{n}{feature\PYZus{}importance}\PY{p}{,} \PY{l+s+s1}{\PYZsq{}}\PY{l+s+s1}{Std Dev}\PY{l+s+s1}{\PYZsq{}}\PY{p}{:} \PY{n}{feature\PYZus{}importance\PYZus{}std}\PY{p}{\PYZcb{}}\PY{p}{)}
    \PY{n}{unimportant\PYZus{}features} \PY{o}{=} \PY{p}{[}\PY{p}{]}
    \PY{n}{feature\PYZus{}importance\PYZus{}df} \PY{o}{=} \PY{n}{feature\PYZus{}importance\PYZus{}df}\PY{o}{.}\PY{n}{sort\PYZus{}values}\PY{p}{(}\PY{n}{by}\PY{o}{=}\PY{l+s+s1}{\PYZsq{}}\PY{l+s+s1}{Importance}\PY{l+s+s1}{\PYZsq{}}\PY{p}{,} \PY{n}{ascending}\PY{o}{=}\PY{k+kc}{False}\PY{p}{)}

    \PY{k}{for} \PY{n}{i} \PY{o+ow}{in} \PY{n+nb}{range}\PY{p}{(}\PY{n+nb}{len}\PY{p}{(}\PY{n}{feature\PYZus{}importance\PYZus{}df}\PY{p}{)}\PY{p}{)}\PY{p}{:}
        \PY{k}{if} \PY{n}{feature\PYZus{}importance\PYZus{}df}\PY{p}{[}\PY{l+s+s1}{\PYZsq{}}\PY{l+s+s1}{Importance}\PY{l+s+s1}{\PYZsq{}}\PY{p}{]}\PY{p}{[}\PY{n}{i}\PY{p}{]} \PY{o}{\PYZlt{}}\PY{o}{=} \PY{l+m+mi}{0}\PY{p}{:}
            \PY{n}{unimportant\PYZus{}features}\PY{o}{.}\PY{n}{append}\PY{p}{(}\PY{n}{feature\PYZus{}importance\PYZus{}df}\PY{p}{[}\PY{l+s+s1}{\PYZsq{}}\PY{l+s+s1}{Feature}\PY{l+s+s1}{\PYZsq{}}\PY{p}{]}\PY{p}{[}\PY{n}{i}\PY{p}{]}\PY{p}{)}

    \PY{c+c1}{\PYZsh{} Plota as importancias}
    \PY{n}{plt}\PY{o}{.}\PY{n}{figure}\PY{p}{(}\PY{n}{figsize}\PY{o}{=}\PY{p}{(}\PY{l+m+mi}{10}\PY{p}{,} \PY{l+m+mi}{6}\PY{p}{)}\PY{p}{)}
    \PY{n}{sns}\PY{o}{.}\PY{n}{barplot}\PY{p}{(}\PY{n}{x}\PY{o}{=}\PY{l+s+s1}{\PYZsq{}}\PY{l+s+s1}{Importance}\PY{l+s+s1}{\PYZsq{}}\PY{p}{,} \PY{n}{y}\PY{o}{=}\PY{l+s+s1}{\PYZsq{}}\PY{l+s+s1}{Feature}\PY{l+s+s1}{\PYZsq{}}\PY{p}{,} \PY{n}{data}\PY{o}{=}\PY{n}{feature\PYZus{}importance\PYZus{}df}\PY{p}{)}
    \PY{n}{plt}\PY{o}{.}\PY{n}{errorbar}\PY{p}{(}\PY{n}{x}\PY{o}{=}\PY{n}{feature\PYZus{}importance\PYZus{}df}\PY{p}{[}\PY{l+s+s1}{\PYZsq{}}\PY{l+s+s1}{Importance}\PY{l+s+s1}{\PYZsq{}}\PY{p}{]}\PY{p}{,} \PY{n}{y}\PY{o}{=}\PY{n}{feature\PYZus{}importance\PYZus{}df}\PY{p}{[}\PY{l+s+s1}{\PYZsq{}}\PY{l+s+s1}{Feature}\PY{l+s+s1}{\PYZsq{}}\PY{p}{]}\PY{p}{,}
                 \PY{n}{xerr}\PY{o}{=}\PY{n}{feature\PYZus{}importance\PYZus{}df}\PY{p}{[}\PY{l+s+s1}{\PYZsq{}}\PY{l+s+s1}{Std Dev}\PY{l+s+s1}{\PYZsq{}}\PY{p}{]}\PY{p}{,} \PY{n}{fmt}\PY{o}{=}\PY{l+s+s1}{\PYZsq{}}\PY{l+s+s1}{o}\PY{l+s+s1}{\PYZsq{}}\PY{p}{)}
    \PY{n}{plt}\PY{o}{.}\PY{n}{xlabel}\PY{p}{(}\PY{l+s+s1}{\PYZsq{}}\PY{l+s+s1}{Importance}\PY{l+s+s1}{\PYZsq{}}\PY{p}{)}
    \PY{n}{plt}\PY{o}{.}\PY{n}{ylabel}\PY{p}{(}\PY{l+s+s1}{\PYZsq{}}\PY{l+s+s1}{Feature}\PY{l+s+s1}{\PYZsq{}}\PY{p}{)}
    \PY{n}{plt}\PY{o}{.}\PY{n}{title}\PY{p}{(}\PY{l+s+s1}{\PYZsq{}}\PY{l+s+s1}{Feature Importance in HistGradientBoostingRegressor}\PY{l+s+s1}{\PYZsq{}}\PY{p}{)}
    \PY{n}{plt}\PY{o}{.}\PY{n}{show}\PY{p}{(}\PY{p}{)}
    
    \PY{n+nb}{print}\PY{p}{(}\PY{n}{unimportant\PYZus{}features}\PY{p}{)}

    
\PY{n}{feature\PYZus{}names} \PY{o}{=} \PY{n}{df\PYZus{}train}\PY{o}{.}\PY{n}{columns}\PY{p}{[}\PY{p}{:}\PY{o}{\PYZhy{}}\PY{l+m+mi}{1}\PY{p}{]}
\PY{n}{plot\PYZus{}feature\PYZus{}importance}\PY{p}{(}\PY{n}{model}\PY{p}{,} \PY{n}{X\PYZus{}train}\PY{p}{,} \PY{n}{y\PYZus{}train}\PY{p}{,} \PY{n}{feature\PYZus{}names}\PY{p}{)}
\end{Verbatim}
\end{tcolorbox}


    % Add a bibliography block to the postdoc
    
    
    
\end{document}
